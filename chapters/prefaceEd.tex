\documentclass[output=paper]{langsci/langscibook} 
\ChapterDOI{10.5281/zenodo.4545027}
\title{Preface} 
\author{{Tra\&Co Group}\affiliation{Johannes Gutenberg University Mainz}}  

\abstract{}

\begin{document}
\SetupAffiliations{mark style=none}
\maketitle

\noindent After the first successful International Congress on Translation, Interpreting and Cognition held at the University of Mendoza, Argentina in 2017, the second conference in this series has been hosted by the Tra\&Co Lab at the Johannes Gutenberg University of Mainz in Germersheim, Germany in 2019. The predictive and explanatory power of studies investigating the translation process has been recognised by many in the field, so it is not surprising that cognitive aspects of the translation process have become central in many research endeavours in Translation and Interpreting Studies in recent years. Interdisciplinary paradigms have been useful in the field for a long time, but even more so in Cognitive Translation and Interpreting Studies. Interdisciplinarity has been useful in overcoming the limits of single disciplines, but also to shed light on hitherto hidden phenomena. The aim of the second International Congress on Translation, Interpreting and Cognition was therefore to call for interdisciplinary multi-method approaches. There were contributions on a range of topics which were held together by a central thread, i.e., by the study of cognitive aspects of translation and interpreting. In particular, there were studies which observed behaviour during translation and interpreting – with a focus on training of future professionals, on language processing more generally and language dominance in particular, on the role of working memory during simultaneous interpreting. In addition, there were studies on how to measure translation competence, on the role of technology in the practice of translation, on interpreting and subtitling, on translation of multimodal media texts, on aspects of ergonomics and usability, on emotions and the role they play in the translation process, on translators' self-concept and psychological factors, on writing in a foreign language and finally also on revision and post-editing. For the present publication, we selected a number of contributions, which showcase the breadth and depth of studies that have been presented at the conference. We are grateful for the general support from the Gutenberg Research College (GRC), the Freundeskreis FTSK, and the JGU Internal University Research Funding. Finally, we are extremely grateful for the many reviewers and their valuable suggestions and feedback to the individual contributions of the present proceedings.\bigskip\\

\noindent Tra\&Co Group\hfill Germersheim, September 2020\\
\noindent
(Silvia Hansen-Schirra, Anne-Kathrin Gros, Silke Gutermuth, Ann-Kathrin Habig, Jean Nitzke, Katharina Oster, Moritz Schaeffer, Anke Tardel, and Janna Vert-hein)
\end{document}
