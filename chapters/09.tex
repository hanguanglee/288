\documentclass[output=paper]{langscibook}
\ChapterDOI{10.5281/zenodo.4545045}

\author{Elena Kokanova\affiliation{Northern (Arctic) Federal University} and Maya Lyutyanskaya\affiliation{Northern (Arctic) Federal University} and  Anna Cherkasova\affiliation{Northern (Arctic) Federal University}}
\title[Eye tracking study of reading for translation and ENG/RUS sight translation]{Eye tracking study of reading for translation and English-Russian sight translation}
\abstract{This paper presents the results of an eye tracking study which compares reading for translation and English-Russian sight translation. The participants of this study included both students and professional interpreters who were asked to read and sight translate two texts from their B language (English) into their A language (Russian). The study revealed significant differences in oculomotor activity during reading and sight translating within a group of students and within a group of professionals. This can be explained by the difference in the efficiency of reading for translation, translation strategy and general translation skills.}

\begin{document}
\renewcommand{\lsChapterFooterSize}{\footnotesize}
\SetupAffiliations{mark style=none}
\maketitle
\section{Introduction} 
The study of oculomotor activity during the reading process has been abundant and focused on various aspects (eye movement characteristics, eye movement control, perceptual span, etc.). Eye tracking studies have mostly focused on readability and processing effort for the given text type and thus on empirical research in neurophysiology \citep{Jakobsen2008, Schnitzer2006, Clifton2016}. Eye tracking has proved to be a powerful tool in scientific research and has recently been used in applied linguistics and translation studies \citep{Hansen-Schirra2016}. It allows identifying the objects of attention with high spatial accuracy and temporal precision. Participants try to fixate their gaze on highly informative elements but each person can choose a different strategy for investigating a stimulus and can change it when presented the same stimulus for the second time. This explains numerous findings in fields such as translation memory, reading for translation, distribution of cognitive effort during translation, etc. \citep{hvelplund2014eye}.  
Sight translation is a form of transposing a written text in the source language into an oral text in the target language. The concept of sight translation is understood differently by researchers. One of the disputed issues concerns the status of this form of translation, whether it is considered as a separate form of interpreting or as a training exercise for other forms of interpreting. Most of the current research supports the idea that the key characteristic features of sight translation include the following:

\begin{itemize}
    \item{time pressure (caused by limited time for text comprehension, minimum time for finding the translation decisions, high speed of speaking);} 
    \item{strict self-control (as self-corrections are not allowed) \citep{Chmiel2013, Kokanova2016,Thawabteh2015}.}
\end{itemize}
In cognitive terms, sight translation is a complex set of brain operations including processing visual input in one language, creating the oral message in another language and control of the translation process at the same time. The actual application of sight translation takes place in a number of professional settings and, despite this fact, seems to be rarely taught as a separate form of interpreting. 

\section{Research design}
The objective of the present research is to collect and compare statistical data on oculomotor activity during reading for translation and English-Russian sight translation by a group of students and a group of professional interpreters.
The hypothesis of the study is that within each group of participants there will be differences between experimental tasks of reading for translation and sight translation, which will allow us to see if professional interpreters demonstrate some kind of translation strategy affecting the result.

\subsection{Participants}
The study was conducted at Northern (Arctic) Federal University, Arkhangelsk, Russia. The first group of participants included eighteen bachelor and master students (average age: 21) with one year of sight translation training. The group of participants included students with B2\slash C1 level of the English language. Command of English was tested before the experiment (\url{https://cambridgeenglish.org}).
The second group consisted of ten participants (average age: 35). All of the participants are professional interpreters working in various fields in Arkhangelsk; the average work experience is 12 years. All participants denied having suffered any brain injuries, neurological conditions, or eyesight pathology and took part in the study voluntarily.

\subsection{Procedure and equipment}
Gaze behaviour of the participants was recorded on the basis of saccades and fixations in the infrared radiation spectrum. For the recording of eye tracking data, the system iView XTM RED (SMI, Germany) for non-contact measurement was used. The collected data were analyzed using BeGaze software. The frequency of the system was 500\,Hz; the viewing distance was 55--60\,cm from the screen. The experiment was conducted in accordance with ethical standards represented in the Declaration of Helsinki (DoH) and European Community directives (8/609 EC).

The participants were asked to read for two minutes and sight translate two texts from their B language (English) into their A language (Russian). Time for translation was not limited. The participants’ translations were recorded for further linguistic analysis and the participants were informed about this.

The texts included abbreviations, position titles, references to historic and cultural events and phenomena such as direct speech, epithets, and metaphors. The dependent variables included measures assumed to indicate cognitive load of lexical units, such as fixation count and saccade count \citep{Kokanova2018}.

\subsection{Coh-Metrix analysis of the source texts}
Both texts were analyzed by the computer tool Coh-Metrix \citep{Graesser2004} using a number of parameters. The first parameter concerned the overall readability of the texts, i.e. their difficulty level. The output of the Flesch reading ease formula is a number from 0 to 100, with a higher score indicating easier reading. Text 1 was assessed as \textit{fairly difficult} to read in accord with the Flesch reading ease formula and was given a score of 51.597. The score for Text 2 was 72.022.

Syntactically, Text 1 was simpler than Text 2 (syntactic simplicity 73.57\% and 53.98\%, respectively). This component reflects how low the number of words are and how simple the syntactic structure is, which is less challenging to process. At the opposite end of the continuum are texts that contain sentences with more words and use complex syntactic structures.

Text 1 contained mostly factual information presented by such language units as abbreviations, position titles, references to historic and cultural events and phenomena. This is confirmed by the concreteness level (92.36\%). Texts containing content words that are concrete, meaningful, and evoke mental images are generally considered to be easier to process and understand. Text 2 was more descriptive and contained such elements as metaphors, epithets, abstract words, so that the concreteness level was lower (53.98\%). Abstract words represent concepts that are considered difficult to visualize. Texts that contain more abstract words are more challenging to understand.

Text 1 was characterised as having a higher connectivity level (71.23\%), the component which reflects the degree to which the text contains explicit adversative, additive, and comparative connectives to express relations in the text. This component reflects the number of logical relations in the text that are explicitly conveyed. This score is likely to be related to the reader’s deeper understanding of the relations in the text. The connectivity level of Text 2 was very low (14.23\%).

\section{Data analysis and results}
The statistical analysis of the parameters under research was carried out using SPSS version 22.0. Data processing included a comprehensive analysis of the normal distribution, and since a number of parameters did not match the Gaussian distribution, the Mann-Whitney \textit{U} test was used to compare the samples. To describe the data, the median (Me) and the first and third quartiles (Q1; Q3) were taken. Differences were considered statistically significant when the probability of erroneous acceptance of the null hypothesis of the absence of differences between samples was $p <0.05$.

We assessed the eye movement parameters for each group of participants separately. Mostly we wanted to see if there were any significant difference between reading for translation and sight translation.

The results for students are presented in Table \ref{tab:1:frequencies}. The data revealed no real difference in fixation and saccade count between Text 1 reading and translation. Total fixation duration was lower during the translation process. Average saccade velocity and average saccade amplitude increased while translating whereas frequency of fixations decreased.

\begin{table}
\caption{Eye tracking measurements for reading and sight translation in the group of students\label{tab:1:frequencies}}
 \begin{tabularx}{\textwidth}{Qrrr rrr r}
 \lsptoprule
    Metric & \multicolumn{3}{c}{Reading Text 1}  & \multicolumn{3}{c}{Translation Text 1} & \multicolumn{1}{c}{$p$}\\\cmidrule(lr){2-4}\cmidrule(lr){5-7}
           & \multicolumn{1}{c}{Мe} & \multicolumn{1}{c}{Q1} & \multicolumn{1}{c}{Q3} & \multicolumn{1}{c}{Мe} & \multicolumn{1}{c}{Q1} & \multicolumn{1}{c}{Q3} & \\
  \midrule
  Fixation count                    & 395.0 & 380.3 & 419.3 & 360.0 & 262.0 & 506.8 & 0.141\\
  Saccade count                     & 399.0 & 351.5 & 407.5 & 383.0 & 271.5 & 550.5 & 0.776\\
  Total fixation duration, sec.     & 96.9  & 93.3 & 98.6   & 80.1  & 66.2  & 103.0  & 0.016\\
  Av. saccade velocity,  degree/sec & 82.4  & 72.4 & 89.5   & 96.9  & 94.1  & 114.8  & 0.002\\
  Fixation frequency  fix/sec       & 3.3   & 3.17 & 3.5    & 2.6   & 2.3   & 3.3     & 0.008\\
  Av. saccade amplitude,  degree    & 3.5   & 3.25 & 4.0    & 4.2   & 3.25  & 4.0    & 0.005\\
  \midrule
   Metric & \multicolumn{3}{c}{Reading Text 2}  & \multicolumn{3}{c}{Translation Text 2} & \multicolumn{1}{c}{$p$}\\\cmidrule(lr){2-4}\cmidrule(lr){5-7}
           & \multicolumn{1}{c}{Мe} & \multicolumn{1}{c}{Q1} & \multicolumn{1}{c}{Q3} & \multicolumn{1}{c}{Мe} & \multicolumn{1}{c}{Q1} & \multicolumn{1}{c}{Q3} & \\
  \midrule
    Fixation count &                    400.5 & 361.0 & 416.8 & 309.0 & 200.3 & 391.5 & 0.014\\
    Saccade count &                     381.0 & 338.0 & 428.5 & 289.0 & 221.0 & 413.0  & 0.018\\
    Total fixation  duration, sec  &    94.6  & 89.5 & 101.9 &   73.9 & 49.8 & 94.8  &    0.011\\
    Av. saccade velocity  degree/sec  & 88.6  & 71.6 & 98.1 &    97.5 & 78.8 & 110.5  &    0.064\\
    Fixation frequency,  fix/sec  &     3.4   & 3.0 & 3.5 &       2.9 & 2.6 & 3.3  &    0.002\\
    Av. saccade amplitude,  degree  &   3.8   & 3.3 & 4 &         4.4 & 3.5 & 5.0 &  0.247\\
   \lspbottomrule
   \end{tabularx}
 \end{table}
 
It was observed for Text 2 that fixation count and saccade count were substantially down during the translation task, compared to Text 1. Total fixation duration and fixation frequency are also on the decline. Average saccade velocity and average saccade amplitude did not show significant changes.

The statistical analysis of eye-tracking parameters in the group of professionals showed some differences between the experimental tasks (Table \ref{tab:2:frequencies}).

\begin{table}[p]
\caption{Eye tracking measurements for reading and sight translation in the group of professionals}
\label{tab:2:frequencies}
 \begin{tabularx}{\textwidth}{Qrrr rrr r}
 \lsptoprule
    Metric & \multicolumn{3}{c}{Reading Text 1}  & \multicolumn{3}{c}{Translation Text 1} & \multicolumn{1}{c}{$p$}\\\cmidrule(lr){2-4}\cmidrule(lr){5-7}
           & \multicolumn{1}{c}{Мe} & \multicolumn{1}{c}{Q1} & \multicolumn{1}{c}{Q3} & \multicolumn{1}{c}{Мe} & \multicolumn{1}{c}{Q1} & \multicolumn{1}{c}{Q3} & \\
  \midrule
  Fixation count                     &   397.0 &   354.5 & 423.5 &    267.5 & 217.8 & 326.6  &    0.010\\
  Saccade count                      &  383.0  &  352.6 & 414.3 &  322.0    &  203.0 & 364.3 &  0.034\\
  Total fixation  duration, sec      &   93.8  &  85.0 & 97.7 &    61.3     & 41.7 & 80.1  &    0.010\\
  Av. saccade velocity,  degree/sec  &   82.7  &  68.8 & 97.8 &   100.0     & 81.6 & 102.4  &    0.174\\
  Fixation frequency,  count/sec     & 3.5     & 3.3 & 3.8 &  2.9           & 2.6 & 3.1 &  0.053\\
  Av. saccade amplitude,  degree     &  4.4    & 4.1 & 5.3 &   4.5          &  4.2 & 5.1 &  0.306\\
  \midrule
   Metric & \multicolumn{3}{c}{Reading Text 2}  & \multicolumn{3}{c}{Translation Text 2} & \multicolumn{1}{c}{$p$}\\\cmidrule(lr){2-4}\cmidrule(lr){5-7}
           & \multicolumn{1}{c}{Мe} & \multicolumn{1}{c}{Q1} & \multicolumn{1}{c}{Q3} & \multicolumn{1}{c}{Мe} & \multicolumn{1}{c}{Q1} & \multicolumn{1}{c}{Q3} & \\
  \midrule
   Fixation count                    &  374.0 & 355.0 & 401.6  &  249.0 &   154.0  & 306.6 &   0.001\\
   Saccade count                     &  387.5 & 361.0 & 439.5  &  232.5 &  186.0   & 321.0 &  0.001\\
   Total fixation  duration, sec     &   83.8 & 78.7  & 96.4   &  57.8  &  32.7    & 74.3  &   0.005\\
   Av. saccade velocity, degree/sec  &   96.4 & 86.4  & 98.3   &  114.2 &  103.7   & 128.2 &    0.131\\
   Fixation frequency, count/sec     & 3.15   & 3.0   & 3.4    &  2.7   & 2.3      & 3.3   &  0.160\\
   Av. saccade amplitude, degree     &  4.9   & 3.9   & 5.9    &   5.2  & 4.3      & 6.7   &  0.570\\
  \lspbottomrule
   \end{tabularx}
 \end{table} 
 
Fixation count and saccade count for sight translation task were lower than for reading task both in Text 1 and Text 2. However, the saccade count the difference between reading and translation tasks was bigger for Text 2. Total fixation duration went down during translation task compared to reading for translation. 

The total fixation duration during reading for translation in the group of students were about 80\% and in the group of professionals were about 70\%. This parameter goes down in the translation task for both groups, although in the group of professionals this difference is bigger.

It should be noted that eye tracking data revealed meaningful differences in fixation and saccade count between reading and sight translation only in Text~2 in the group of students. From the noticeable decrease of fixation count (from 400.5 to 309.0) and saccade count (from 381.0 to 289.0) in Test 2 it can be assumed that there are some factors making translation of the second text easier for students. This may have been an effect of the warming-up period. Also, after finishing Text~1 the participants were more adapted to the stressful situation and, as both texts have the same subject matter, the general context could become a supporting factor.

As for the group of professional interpreters, fixation count and saccade count decreased in the translation tasks for both texts. There were certain stability in oculomotor behaviour of professional interpreters when performing experimental tasks. As fixations are the period of time when the eyes remain fairly still and new information is acquired from the visual array, and saccades search for new meaningful areas of fixation \citep{Rayner2009}, supposedly, this shows that professional interpreters demonstrated some strategy in analyzing the context and searching for translation equivalents while reading the text.

This leads to the assumption that professional interpreters do the quicker search for key support words in the source text during sight translation. There is a clear-cut difference between the translation time of Text 1 and Text 2. In the group of students the average translation time for Text 1 was 2 min 12 sec and for Text 2 it was 1 min 47 sec. In the group of professionals the average translation time for Text 1 was 1 min 39 sec, and for Text 2 it was 1 min 16 sec.

In the student group the frequency of fixations during translation was lower than during the reading task. Translation of Text 1 shows an increase in the average saccade amplitude and velocity. Translation of Text 2 indicates a decrease in the fixation and saccade count. Supposedly, this shows the quicker search for key support words in the source text in a stressful situation like sight translation.

\section{Conclusion and further research}
The eye tracking data seem to support the hypothesis of the present study as professional participants did demonstrate significantly lower fixation and saccade count between reading and translation tasks. The meaningful difference in fixation and saccade counts between reading and translation tasks in the students’ group was observed only in Text 2. The research has shown that English-Russian sight translation can cause difficulties for students because of the low level of silent reading skills. Oculomotor behaviour of professional interpreters is more stable. They seem to reduce their search activity in the form of fixation and saccade count during the sight translation task.

Prospects for further research can include a longitude eye-tracking study of reading, reading for translation and English-Russian sight translation from beginners to semi-professionals based on a more thorough selection of texts using special computer tools for text parameters analysis and introduction of reading for translation training into the interpreting course. The results of further research can be used to work out recommendation for students on how to use the reading time more efficiently, how not to miss key elements in the text, how to overcome garden-path sentences and so on.

An interdisciplinary approach in translation studies can shed more light on translation as a decision making process and provide teachers with more tools for improving students’ professional skills.
    

{\sloppy\printbibliography[heading=subbibliography,notkeyword=this]}
\end{document}
